% % LLNCS macro package for Springer Computer Science proceedings;
% Version 2.21 of 2022/01/12
%
\documentclass[runningheads]{llncs}
%
\usepackage[T1]{fontenc}
% T1 fonts will be used to generate the final print and online PDFs,
% so please use T1 fonts in your manuscript whenever possible.
% Other font encondings may result in incorrect characters.
%

\usepackage{amsmath}
\usepackage{todonotes}

\usepackage{graphicx, color, float, subfig}
\captionsetup{font=small,labelfont=bf,labelsep=period,skip=5pt}
\captionsetup[subfloat]{font=scriptsize,labelfont=bf,skip=5pt}

%
\usepackage{hyperref}
\renewcommand\UrlFont{\color{blue}\rmfamily}
\urlstyle{rm}
%

\begin{document}
\title{\fontsize{12}{12}\selectfont Project 1 - Cat \& Dog Classification}
\titlerunning{Project 1}
%
\author{Henrik Daniel Christensen\orcidID{hench13@student.sdu.dk} \\Frode Engtoft Johansen\orcidID{fjoha21@student.sdu.dk}}
\authorrunning{Christensen, Johansen} % first names are abbreviated in the running head.
%
\institute{University of Southern Denmark, SDU\\\textit{Department of Mathematics and Computer Science}}
%
\maketitle % typeset the header of the contribution

%%%%%%%%%%%%%%%%%%%%%%%%%%%%%%%%%%%%%%%%%%%
\section{Introduction}
The objective of this project is to develop a deep learning model capable of distinguishing between images of cats and dogs.
The task involves training a neural network using a dataset containing 3,600 images, equally divided between the two categories. The code and the notebook including all the results is available in the Appendix.

% The report is structured as follows: Section 2 explores the dataset and the possible features that can be extracted from it.
% Section 3 presents the architecture of the base model.
% Section 4 describes which the regularization techniques used, including data augmentation.
% Section 5 describes the final model for predicting cats and dogs and presents the results.
% Section 6 presents the results of a pretrained model 'AlexNet'.
% Finally, Section 7 concludes the report.
% In the Appendix, the code and a notebook with the results and visualizations are provided.
%%%%%%%%%%%%%%%%%%%%%%%%%%%%%%%%%%%%%%%%%%%

%%%%%%%%%%%%%%%%%%%%%%%%%%%%%%%%%%%%%%%%%%%
%\input{base/credits}


\bibliographystyle{base/splncs04}
\newpage
\bibliography{base/references}
%%%%%%%%%%%%%%%%%%%%%%%%%%%%%%%%%%%%%%%%%%%
\end{document}