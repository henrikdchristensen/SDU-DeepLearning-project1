% % LLNCS macro package for Springer Computer Science proceedings;
% Version 2.21 of 2022/01/12
%
\documentclass[runningheads]{llncs}
%
\usepackage[T1]{fontenc}
% T1 fonts will be used to generate the final print and online PDFs,
% so please use T1 fonts in your manuscript whenever possible.
% Other font encondings may result in incorrect characters.
%

\usepackage{amsmath}
\usepackage{todonotes}

\usepackage[left=2cm,
            right=2cm,
            top=2cm,
            bottom=2cm]{geometry}

\usepackage{graphicx, color, float, subfig}
\captionsetup{font=small,labelfont=bf,labelsep=period,skip=5pt}
\captionsetup[subfloat]{font=scriptsize,labelfont=bf,skip=5pt}

%
\usepackage{hyperref}
\renewcommand\UrlFont{\color{blue}\rmfamily}
\urlstyle{rm}
%

\begin{document}
\title{\fontsize{12}{12}\selectfont Project 1 - Cat \& Dog Classification}
\titlerunning{Project 1}
%
\author{Henrik Daniel Christensen\orcidID{hench13@student.sdu.dk} \\Frode Engtoft Johansen\orcidID{fjoha21@student.sdu.dk}}
\authorrunning{Christensen, Johansen} % first names are abbreviated in the running head.
%
\institute{DM873: Deep Learning\\University of Southern Denmark, SDU\\\textit{Department of Mathematics and Computer Science}}
%
\maketitle % typeset the header of the contribution

%%%%%%%%%%%%%%%%%%%%%%%%%%%%%%%%%%%%%%%%%%%
\section{Introduction}
The objective of this project is to develop a deep learning model capable of distinguishing between images of cats and dogs.
The task involves training a neural network using a dataset containing 3,600 images, equally divided between the two categories. The code and the notebook including all the results is available in the Appendix.

% The report is structured as follows: Section 2 explores the dataset and the possible features that can be extracted from it.
% Section 3 presents the architecture of the base model.
% Section 4 describes which the regularization techniques used, including data augmentation.
% Section 5 describes the final model for predicting cats and dogs and presents the results.
% Section 6 presents the results of a pretrained model 'AlexNet'.
% Finally, Section 7 concludes the report.
% In the Appendix, the code and a notebook with the results and visualizations are provided.
\section{Explorative Analysis}
The dataset contains a mix of different breeds of cats and dogs.
The images are very diverse in terms of size, orientation, lighting, focus, and resolution.
Also, the images are taken from different angles and distances.
The images are mostly around 200-500 pixels width/height and are all in color, meaning they have 3 color channels.
In Figure \ref{fig:cats_dogs}, examples of cats and dogs from the dataset are shown.
\begin{figure}[H]
    \vspace*{-0.7cm}
    \centering
    \subfloat[][Cats]{%
        \includegraphics[width=0.48\textwidth]{figures/cats.png}\label{fig:cats}}\hspace{0.4cm}
    \subfloat[][Dogs]{%
        \includegraphics[width=0.48\textwidth]{figures/dogs.png}\label{fig:dogs}}
    \caption{Sample images of cats and dogs from the dataset.}
    \label{fig:cats_dogs}
    \vspace*{-0.7cm}
\end{figure}

From the images, clearly, cats and dogs have different features that can be used to distinguish between them.
For example, cats have generally shorter faces often with triangular ears and pointed noses,
while dogs generally have longer snouts and have more different ear shapes.
Moreover, the eyes of cats are generally more sharp-shaped, while dogs have rounder eyes. The goal is therefore to create a model that can learn some of these features.
\section{Base Model}
First, we set out to develop a base model. However, first, we had to decide which image size to use.
We chose an image size of 224 by 224, since almost all images are bigger on each axis, so we almost only shrink pictures.
The pretrained models also uses resizing to 224 pixels making our model more comparable to the pretrained ones.
Resizing the images has the effect of reducing the size of the input to the model which will reduce the accuracy a little bit, but make it train significantly faster.

Next, we had to decide how many layers our model should have.
As the task is relatively simple, we decided to use 4 convolutional layers and 3 fully connected layers.
The base model is shown in Table \ref{tab:base_model}.
\begin{table}[H]
    \vspace*{-0.5cm}
    \centering
    \begin{tabular}{|l|c|c|c|c|}
    \hline
                & \textbf{Output}           & \textbf{Kernel}   & \textbf{MaxPooling}   & \textbf{Activation}   \\ 
                & \textbf{kernels/features} &                   &   &   \\ \hline
    Conv2D w/   & 32                        & 3x3                   & 2x2                   & ReLU                  \\ \hline
    Conv2D w/   & 64                        & 3x3                   & 2x2                   & ReLU                  \\ \hline
    Conv2D w/   & 128                       & 3x3                   & 2x2                   & ReLU                  \\ \hline
    Conv2D w/   & 256                       & 3x3                   & 2x2                   & ReLU                  \\ \hline
    Linear w/   & 256                       & -                     & -                     & ReLU                  \\ \hline
    Linear w/   & 128                       & -                     & -                     & ReLU                  \\ \hline
    Linear w/   & 2                         & -                     & -                     & -                     \\ \hline
    \end{tabular}
    \caption{Base Model.}
    \label{tab:base_model}
    \vspace*{-0.8cm}
\end{table}
Results of the base model after 25 epochs are shown in Figure \ref{fig:base_model_results}.
% \begin{figure}[H]
    %     \vspace*{-0.7cm}
    %     \centering
    %     \includegraphics[width=0.8\textwidth]{figures/pretrained_model_results.png}
    %     \caption{Base Model Results.}
    %     \label{fig:base_model_results}
    %     \vspace*{-0.7cm}
    % \end{figure}

Clearly, the base model is overfitting, one way, especially for small datasets as in this case, to reduce overfitting is to use data augmentation, enabling the model to learn from more data.
\section{Regularization}

\subsection{Data Augmentation}
One way, especially for small datasets as in this case, to reduce overfitting is to use data augmentation, enabling the model to learn from more data.

We chose to augment the data in a few different ways since we want to generalize the data so the important features remain. 
Using augmentation ensures that it will get better accuracy when faced with new images. In other words, it reduces overfitting.

The images are of varying quality and taken in different lighting, and this is also what we can expect from the final test set.
For this reason we change the brightness, saturation, and contrast. Furthermore, the pictures are taken from different angles so we rotate the pictures slightly and flip them horizontally. Generally, images of animals are not vertically flipped, so we chose not to do this, but of course, some cases may exist.

The pictures are also cropped differently to get finer details from the images, as well as the images becomes scale invariant.

Even though some of the pictures are more blurry than others, we chose to not use blurring as data augmentation, as it effectively just reduces the data of the image without reducing input to the model.

We shear and translate the pictures to emulate pictures taken from different angles of the animal.

Since the training dataset is quite small, we wanted to create more data by reusing the same images but with random augmentations, but in the end we found out that our results did not improve from this.


Many different data augmentation techniques exists, to find the best data augmentation techniques for this task, one image was used to test different techniques.



The techniques used for this task as well as augmented sample images by these techniques are shown in Figure \ref{fig:augmentation}.

\begin{figure}[H]
    \vspace*{-0.7cm}
    \centering
    \subfloat[Data Augmentation Techniques.\label{tab:augmentation}]{
        \raisebox{\height}{ % align at the bottom
        \begin{tabular}{|l|p{2.8cm}|}
            \hline
            \textbf{Data Augmentation} & \textbf{Parameters} \\ \hline
            RandomHorizontalFlip & 50\% \\ \hline
            ColorJitter & brightness=0.2 \newline contrast=0.2 \newline saturation=0.2 \newline hue=0 \\ \hline
            RandomAffine & degrees=25 \newline translate=(0.1, 0.1) \newline scale=(0.7, 1.3) \newline shear=(-10, 10) \\ \hline
        \end{tabular}}}
    \hspace{0.4cm}
    \subfloat[Sample images after augmentation.\label{fig:augmentation_images}]{\includegraphics[width=0.5\textwidth]{figures/augmentation.png}}
    \caption{Data augmentation techniques.}
    \label{fig:augmentation}
    \vspace*{-0.7cm}
\end{figure}


\begin{figure}[H]
    \centering
    \includegraphics[width=0.4\textwidth]{figures/results_augmentation.png}
    \caption{Sample images after augmentation.}
    \label{fig:augmentation_results}
\end{figure}


\subsection{More Regularization Techniques}


1:
- 93% valideringsæt
- 84% test

2:
- 93% valideringsæt
- 85% test sæt

- loss mere stabil på 2. Den er med weight decay og halvering af learning rate hver 25 epoker.
- dropout layer

\begin{figure}[H]
    \vspace*{-0.7cm}
    \centering
    \subfloat[Regularized model 1.\label{fig:reg1}]{\includegraphics[width=0.4\textwidth]{figures/results_reg_1.png}}
    \hspace{1cm}
    \subfloat[Regularized model 2.\label{fig:reg2}]{\includegraphics[width=0.4\textwidth]{figures/results_reg_2.png}}
    \caption{Results using regularization.}
    \label{fig:reg}
    \vspace*{-0.7cm}
\end{figure}

\subsection{Adding one more convolutional layer}
\begin{figure}[H]
    \vspace*{-0.7cm}
    \centering
    \subfloat[Regularized model 3.\label{fig:reg1}]{\includegraphics[width=0.4\textwidth]{figures/results_reg_3.png}}
    \hspace{1cm}
    \subfloat[Regularized model 4.\label{fig:reg2}]{\includegraphics[width=0.4\textwidth]{figures/results_reg_4.png}}
    \caption{Results using one more convolutional layer.}
    \label{fig:reg}
    \vspace*{-0.7cm}
\end{figure}

\#TODO: Tilføj feature maps
\section{Prediction}
For prediction on the test set, we choose the model 
\section{Pretrained Model}
To see if we could improve the model, a pretrained AlexNet model is implemented.
This model is pretrained on the ImageNet dataset, which contains 1.2 million images and 1000 classes, also including cats and dogs.
However, to use the pretrained model, the output layer is modified to fit the binary classification task of cats and dogs.

The results of the pretrained model after 25 epochs are shown in Figure \ref{fig:pretrained_model_results}.

% \begin{figure}[H]
%     \vspace*{-0.7cm}
%     \centering
%     \includegraphics[width=0.8\textwidth]{figures/pretrained_model_results.png}
%     \caption{Pretrained Model Results.}
%     \label{fig:pretrained_model_results}
%     \vspace*{-0.7cm}
% \end{figure}

\section{Conclusion}
We ended up with a model that has ?? results. The model we used was ?? with ?? layers having XX parameters. This result is pretty good, but it could be improved. It is worth noticing that some of the pictures are very hard to classify, like dog.1335 and dog.1395.

- compare results to pretrained model - why is the pretrained model better?


The model could have been improved in various ways.
First of all more training data would very likely improve the model a lot. In some of the pictures its very hard to see whether its a cat or a dog and its doubtful that the model would learn anything from the picture, so removing those pictures from the dataset would potentially help the model learn more meaningful features and also train faster.
We could have run epochs until the model began overfitting and then stopped at that point. If time was not a factor this is probably what we would have done.
We could have also experimented more with the different parameters like learning rate and weight decay. 
We could have tried even more data augmentations to see what works best. We only rotate the pictures up to 25 degrees, which means that the model as an example would have a hard time recognising pictures of animals upside down.


\subsection{Individual Contributions}
\begin{table}[H]
    \centering
    \begin{tabular}{|l|p{5cm}|p{5cm}|}
    \hline
                    & \textbf{Henrik Daniel Christensen} & \textbf{Frode Engtoft Johansen} \\ \hline
    \textbf{Code}   & - Base model \newline - Data Augmentation \newline - Regularization \newline - Pre-trained & \\ \hline
    \textbf{Report} & - Introduction \newline - Explorative Analysis \newline - Base Model \newline - Regularization \newline - Prediction \newline - Pretrained Model \newline - Conclusion & \\ \hline
    \end{tabular}
    \caption{Individual contributions.}
    \label{tab:individual_contributions}
\end{table}


%%%%%%%%%%%%%%%%%%%%%%%%%%%%%%%%%%%%%%%%%%%

%Appendices
\appendix
\input{appendices/a_code}
\input{appendices/b_notebook}

% \bibliographystyle{base/splncs04}
% \newpage
% \bibliography{base/references}
%%%%%%%%%%%%%%%%%%%%%%%%%%%%%%%%%%%%%%%%%%%
\end{document}